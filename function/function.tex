\documentclass[preview,multi,crop=false,border=1in,class=memoir]{standalone}
\usepackage{../common}

\begin{document}
\begin{preview-page}
\section{函数}

Erlang中最最基本的两个概念就是函数和模式了。其实,算术运算也会先转换成
函数调用,再计算。比如,初识Erlang中最开始的例子可以写成下面这样。

\begin{ErlangShellSession}
1> erlang:'+'(1,1).
2
2> erlang:'-'(5,2).
3
3> erlang:'*'(2,2).
4
4> erlang:'+'(1,erlang:'*'(2,2)).
5
5> erlang:'*'(erlang:'+'(1,2),2).
6
6>
\end{ErlangShellSession}

注意到 \verb|+| 两边都有 \verb|'| 。通常 \verb|atom()| 都是由小写字母开
头的。Erlang也允许非小写字母开头或者有特殊字符的 \verb|atom()| 。这些
\verb|atom()| 就需要前后都有 \verb|'| 了。当然了,在小写字母开头的
\verb|atom()| 两边都加上 \verb|'| 会被认为是和不加相同的 \verb|atom()| 。

\begin{ErlangShellSession}
1> a = 'a'.
a
2>
\end{ErlangShellSession}

现在就来看一下怎么解释函数。


\subsection{Environment}

函数中定义的变量,在调用的过程中,会有对应的值。比如,

把这种对应称为Environment。要解释函数,就要能表示Environment。方便起见,
这里先用小写字母开头的 \verb|atom()| 来表示变量名。

\begin{Exercise}[title={Environment},difficulty=2]
\verb|environ|

\verb|new/0|
\verb|bind/2|
\verb|is_bound/1|
\verb|get_value/1|
\end{Exercise}

\nonzeroparskip

我们可以用一个特殊的 \verb|list()| 来表示Environment。这个
\verb|list()| 的每个元素都是一个特殊的 \verb|tuple()| 。这个 \verb|tuple()|
的第一个元素用来表示变量名,第二个元素用来表示变量的值。


\verb|subst/2|


\subsection{S表达式}

定义S表达式 \cite{McCarthy:1960:symbolic}

\begin{itemize}
\item 若 \verb|Expr| 是一个 \verb|atom()| , \verb|Expr| 是一个S表达式。
\item 若 \verb|Expr| 是一个 \verb|list()| ,且 \verb|Expr| 里的所有元素都是S表达式,那么 \verb|Expr| 是一个S表达式。
\end{itemize}

\begin{Exercise}[difficulty=1]
\verb|is_sexpr/1|
\end{Exercise}

\nonzeroparskip

若S表达式是一个 \verb|atom()| ,将其视为变量名,其值可以在Environment中
代换得到。若一个S表达式是一个 \verb|list()| ,我们把它当作是一个函数调
用,其中,这个 \verb|list()| 第一个元素的值,是被调用的函数,剩余的元素
就是传入的参数。

\subsection{quote, label}

\subsection{car, cdr, cons}

\subsection{atom, eq, cond}

\subsection{lambda}



\end{preview-page}
\end{document}
