\documentclass[preview,multi,crop=false,border=1in,class=memoir]{standalone}
\usepackage{../common}

\begin{document}
\begin{preview-page}
\section{函数}

函数是Erlang里最基本的概念之一。其实,算术运算也会先转换成函数调用再计
算。比如,初识Erlang中最开始的例子可以写成下面这样。

\begin{ErlangShellSession}
1> erlang:'+'(1,1).
2
2> erlang:'-'(5,2).
3
3> erlang:'*'(2,2).
4
4> erlang:'+'(1,erlang:'*'(2,2)).
5
5> erlang:'*'(erlang:'+'(1,2),2).
6
6>
\end{ErlangShellSession}


现在就来看一下怎么解释函数。


\subsection{Environment}

函数中定义的变量,在调用的过程中,会有对应的值。比如,



把这种对应关系称为Environment。要解释函数,就要能表示Environment。方便起见,
这里先用小写字母开头的 \verb|atom()| 来代表变量名。

\begin{Exercise}[title={environ},difficulty=2]
在 \verb|environ| 模块里定义四个函数,\verb|empty/0|, \verb|is_bound/2|,
\verb|bind/3|, \verb|get_value/2|。

调用 \verb|empty/0| 可以得到一个空的Environment。\verb|is_bound/2| 用来
检查变量名是否在Environment里有定义。\verb|bind/3| 用来把变量名和值关联
起来。\verb|get_value/2| 用来获取变量的值。

\begin{SourceCode}[5][11]{exercise.erl}
check(environ) ->
    Env = environ:empty(),
    false = environ:is_bound(a, Env),
    Env1 = environ:bind(a, b, Env),
    true = environ:is_bound(a, Env1),
    b = environ:get_value(a, Env1),
    ok;
\end{SourceCode}
\end{Exercise}

\begin{Answer}
\begin{SourceCode}[6]{environ.erl}
empty() ->
    [].

bind(Name, Value, Env) ->
    [{Name,Value}|Env].

is_bound(_, []) ->
    false;
is_bound(Name, [{Name, _}|_]) ->
    true;
is_bound(Name, [_|Env]) ->
    is_bound(Name, Env).

get_value(Name, [{Name, Value}|_]) ->
    Value;
get_value(Name, [_|Env]) ->
    get_value(Name, Env).
\end{SourceCode}
\end{Answer}

\nonzeroparskip

现在用一个特殊的 \verb|list()| 来表示Environment。这个 \verb|list()| 的
每个元素都是一个特殊的 \verb|tuple()| 。这个 \verb|tuple()| 的第一个元
素用来表示变量名,第二个元素用来表示变量的值。定义 \verb|subst/2| ,用
来从这样的Environment里取出变量的值。

\begin{SourceCode}[7][12]{func1.erl}
subst(_, []) ->
    none;
subst(K, [{K, V}|_]) ->
    {ok, V};
subst(K, [_|T]) ->
    subst(K, T).
\end{SourceCode}


\subsection{S表达式}

定义S表达式 \cite{McCarthy:1960:symbolic}

\begin{itemize}
\item 若 \verb|Expr| 是一个 \verb|atom()| , \verb|Expr| 是一个S表达式。
\item 若 \verb|Expr| 是一个 \verb|list()| ,且 \verb|Expr| 里的所有元素
都是S表达式,那么 \verb|Expr| 是一个S表达式。
\end{itemize}

\begin{Exercise}[title={sexpr},difficulty=1]
在 \verb|sexpr| 模块里定义 \verb|is_sexpr/1| 函数,判断是否是S表达式。


\begin{SourceCode}[12][20]{exercise.erl}
check(sexpr) ->
    true = sexpr:is_sexpr(a),
    false = sexpr:is_sexpr(1),
    true = sexpr:is_sexpr([]),
    true = sexpr:is_sexpr([a]),
    true = sexpr:is_sexpr([a, b, c]),
    false = sexpr:is_sexpr([1]),
    false = sexpr:is_sexpr([{a,a}]),
    ok;
\end{SourceCode}
\end{Exercise}

\begin{Answer}
\begin{SourceCode}[6][16]{sexpr.erl}
is_sexpr([]) ->
    true;
is_sexpr([Head|Tail]) ->
    case is_sexpr(Head) of
        false ->
            false;
        true ->
            is_sexpr(Tail)
    end;
is_sexpr(Expr) ->
    is_atom(Expr).
\end{SourceCode}
\end{Answer}

\nonzeroparskip

若S表达式是一个 \verb|atom()| ,将其视为变量名,其值可以在Environment中
找到。若一个S表达式是一个 \verb|list()| ,我们把它当作是一个函数调用,
其中,这个 \verb|list()| 第一个元素的值,是被调用的函数,剩余的元素就是
传入的参数。这种求值的规则可以用 \verb|apply/2| 来表示。
\verb|apply/2| 定义如下

\begin{SourceCode}[15][21]{func1.erl}
apply(Expr, Env)
  when is_atom(Expr) ->
    {ok, Value} = subst(Expr, Env),
    {Value, Env};
apply([H|T], Env) ->
    {Fun, Env1} = apply(H, Env),
    call(Fun, T, Env1).
\end{SourceCode}

注意,Erlang编译时默认导入的函数中,也有叫 \verb|apply/2| 的。可以在文
件开头取消。

\begin{SourceCode}[2][2]{func1.erl}
-compile({no_auto_import, [apply/2]}).
\end{SourceCode}

因为现在普通的数据和函数都是从同一个Environment里取出来的,为了区分,用
\verb|{data, X}| 表示普通数据, 用 \verb|{fn, X}| 表示函数。光有一个空
的Environment和求值规则,啥也做不了。接下来就来定义一些基本函数。

\subsection{quote}

假如,我们想让一个表达式的求值结果是一个 \verb|atom()| ,比如 \verb|a|
。因为在求值过程中, 一个 \verb|atom()| 会当作变量名,直接对 \verb|a|
求值得到的会是变量 \verb|a| 的值,而不是 \verb|a| 。为了能得到
\verb|a| ,需要定义 \verb|quote| 函数。这个函数只接受一个参数,调用这个
函数的结果就是传入的参数。

\begin{SourceCode}[32][36]{func1.erl}
test(quote) ->
    {{data, a}, _} =
        apply([quote, a], new_env()),
    {{data, [a,b,c]}, _} =
        apply([quote, [a,b,c]], new_env()).
\end{SourceCode}

先定义\verb|new_env/0| ,用来得到一开始的 Environment。

\begin{SourceCode}[28][29]{func1.erl}
new_env() ->
    [{quote, {fn, quote}}].
\end{SourceCode}

\verb|quote| 的定义很简单。

\begin{SourceCode}[24][25]{func1.erl}
call({fn, quote}, [X], Env) ->
    {{data, X}, Env}.
\end{SourceCode}

S表达式的 \verb|[quote, a]| 作用相当于 Erlang 里的 \verb|a| 。这已经不
是一般的函数了,不过这里仍然把它称作函数。在Erlang里调用一个函数前,会
先对各个参数的表达式求值。定义 \verb|id/1| 函数

\begin{SourceCode}[32][33]{func2.erl}
id(X) ->
    X.
\end{SourceCode}

不难看出

\begin{SourceCode}[36][38]{func2.erl}
test(id) ->
    a = id(a),
    a = id(id(a));
\end{SourceCode}

而在S表达式里,在调用函数前,没有先对表达式求值。所以,
\verb|[quote, [quote, a]]| 的求值结果会是 \verb|[quote, a]| 。

\begin{SourceCode}[39][43]{func2.erl}
test(quote) ->
    {{data, a}, _} =
        apply([quote, a], new_env()),
    {{data, [quote, a]}, _} =
        apply([quote, [quote, a]], new_env()).
\end{SourceCode}


\begin{Exercise}[title={id},difficulty=1]
定义S表达式函数 \verb|id| ,使其和Erlang里 \verb|id/1| 有相同的作用

\begin{SourceCode}[21][26]{exercise.erl}
check(id) ->
    {{data, a}, _} =
        id:apply([id, [quote, a]], id:new_env()),
    {{data, a}, _} =
        id:apply([id, [id, [quote, a]]], id:new_env()),
    ok.
\end{SourceCode}

\end{Exercise}

\begin{Answer}
\begin{SourceCode}[24]{id.erl}
call({fn, quote}, [X], Env) ->
    {{data, X}, Env};
call({fn, id}, [X], Env) ->
    apply(X, Env).


new_env() ->
    [{quote, {fn, quote}},
     {id,    {fn, id}}].
\end{SourceCode}
\end{Answer}


\subsection{label}

定义一个S表达式函数 \verb|label| ,用来指定变量的值。现在不考虑模式,只
要做到和Erlang里 \verb|X = a| 类似就行了。因为要检查后面的表达式里变量
的值, 在定义 \verb|label| 之前,先定义一个辅助函数\verb|eval_list/2|
,用来对一连串S表达式分别求值。

\begin{SourceCode}[36][41]{func3.erl}
eval_list([], Env) ->
    {[], Env};
eval_list([H|T], Env) ->
    {VH, Env1} = apply(H, Env),
    {VT, Env2} = eval_list(T, Env1),
    {[VH|VT], Env2}.
\end{SourceCode}

接着来定义 \verb|label|

\begin{SourceCode}[26][28]{func3.erl}
call({fn, label}, [X,Y], Env) ->
    {Y1, Env1} = apply(Y, Env),
    {Y1, [{X, Y1}|Env1]}.
\end{SourceCode}

这样定义的问题是,前一次指定的值会被后一次指定的值覆盖

\begin{SourceCode}[49][62]{func3.erl}
test(label) ->
    {[{data, a}, {data, a}], _} =
        eval_list([[label, x, [quote, a]],
                   x
                  ], new_env()),
    {[{data, b}, {data, b}], _} =
        eval_list([[label, x, [quote, b]],
                   x
                  ], new_env()),
    {[{data, a}, {data, b}, {data, b}], _} =
        eval_list([[label, x, [quote, a]],
                   [label, x, [quote, b]],
                   x
                  ], new_env()).
\end{SourceCode}

而在Erlang里,这样会导致出错

\begin{ErlangShellSession}
1> X = a.
a
2> X = b.
** exception error: no match of right hand side value b
3>
\end{ErlangShellSession}

不过,这实际上是模式匹配的作用。这里就先保留当前的定义,小心一点不重复
指定同一个变量的值就好了。

\subsection{car, cdr, cons}

因为S表达式由 \verb|list()| 和 \verb|atom()| 组成,所以肯定要定义一些
\verb|list()| 相关的函数。 \verb|car|, \verb|cdr|, \verb|cons| 分别对应
Erlang里的 \verb|car/1|, \verb|cdr/1| 和 \verb|cons/2| 。

\begin{SourceCode}[57][59]{func4.erl}
car([H|_]) -> H.
cdr([_|T]) -> T.
cons(H, T) -> [H|T].
\end{SourceCode}

\verb|car/1| 用来取出一个 \verb|list()| 的第一个元素

\begin{SourceCode}[96][98]{func4.erl}
test(car) ->
    a = car([a]),
    a = car([a,b]);
\end{SourceCode}

\verb|cdr/1| 用来取出一个 \verb|list()| 的除第一个元素以外的所有元素

\begin{SourceCode}[99][101]{func4.erl}
test(cdr) ->
    [] = cdr([a]),
    [b] = cdr([a,b]);
\end{SourceCode}

\verb|cons/2| 用来在一个 \verb|list()| 之前加上一个元素。

\begin{SourceCode}[102][104]{func4.erl}
test(cons) ->
    [a] = cons(a, []),
    [a,b] = cons(a, [b]).
\end{SourceCode}

定义 \verb|car|, \verb|cdr|, \verb|cons| 。

\begin{SourceCode}[35][38]{func4.erl}
call({fn, cons}, [X,Y], Env) ->
    {[{data,X1}, {data,Y1}], Env1} =
        eval_list([X,Y], Env),
    {{data, [X1|Y1]}, Env1}.
\end{SourceCode}

可以看到和Erlang里对应的函数作用是一样的。

\begin{SourceCode}[81][95]{func4.erl}
test(list) ->
    {{data, a}, _} =
        apply([car, [quote, [a]]], new_env()),
    {{data, a}, _} =
        apply([car, [quote, [a,b]]], new_env()),
    {{data, []}, _} =
        apply([cdr, [quote, [a]]], new_env()),
    {{data, [b]}, _} =
        apply([cdr, [quote, [a,b]]], new_env()),
    {{data, [a]}, _} =
        apply([cons, [quote, a], [quote, []]],
              new_env()),
    {{data, [a,b]}, _} =
        apply([cons, [quote, a], [quote, [b]]],
              new_env());
\end{SourceCode}

\subsection{cond}

Erlang里,一个函数可以有多个分句。这也可以用case表达式来表示。因为不考
虑模式,所以这里定义的 \verb|cond| 就相当于只剩下guard的case表达式。 函
数 \verb|cond| 只接受一个 \verb|list()| 类型的参数,其中每个元素都是一
个有两个元素的 \verb|list()| 。逐个对其第一个元素求值,直到找到一个
\verb|list()| ,对其第一个元素求值的结果是 \verb|true| ,那么调用
\verb|cond| 函数的结果就是对其第二个元素求值的结果。

\begin{SourceCode}[100][127]{func5.erl}
test('cond') ->
    {{data, a}, _} =
        apply(['cond',
               [[[quote, true],
                 [quote, a]],
                [[quote, false],
                 [quote, b]],
                [[quote, false],
                 [quote, c]]
               ]], new_env()),
    {{data, b}, _} =
        apply(['cond',
               [[[quote, false],
                 [quote, a]],
                [[quote, true],
                 [quote, b]],
                [[quote, false],
                 [quote, c]]
               ]], new_env()),
    {{data, c}, _} =
        apply(['cond',
               [[[quote, false],
                 [quote, a]],
                [[quote, false],
                 [quote, b]],
                [[quote, true],
                 [quote, c]]
               ]], new_env()).
\end{SourceCode}

以下是 \verb|cond| 的定义

\begin{SourceCode}[39][46]{func5.erl}
call({fn, 'cond'}, [[[P,E]|T]], Env) ->
    {{data, R}, Env1} = apply(P, Env),
    case R of
        false ->
            call({fn, 'cond'}, [T], Env1);
        true ->
            apply(E, Env1)
    end.
\end{SourceCode}

\subsection{atom, eq}

光有这样的 \verb|cond| 并没有什么用,还需要定义一些函数能对值进行一些判
断。先来看 \verb|atom| 。 \verb|atom| 的作用和Erlang里
\verb|is_atom/1| 一样。

\begin{SourceCode}[146][150]{func6.erl}
test(atom) ->
    {{data, true}, _} =
        apply([atom, [quote, a]], new_env()),
    {{data, false}, _} =
        apply([atom, [quote, [a,b,c]]], new_env());
\end{SourceCode}

\verb|atom| 的定义很简单。

\begin{SourceCode}[47][49]{func6.erl}
call({fn, atom}, [X], Env) ->
    {{data, X1}, Env1} = apply(X, Env),
    {{data, is_atom(X1)}, Env1};
\end{SourceCode}

再来看 \verb|eq| 。 \verb|eq| 用来比较两个值是否相同。不过有个例外,对
于两个非空的 \verb|list()| ,即便所有元素都相同,在这里也认为不同。因为
能判断的话,已经具有部分模式匹配的功能了,这是现在需要尽量避免的。

\begin{SourceCode}[151][169]{func6.erl}
test(eq) ->
    {{data, true}, _} =
        apply([eq, [quote, []], [quote, []]],
              new_env()),
    {{data, true}, _} =
        apply([eq, [quote, a], [quote, a]],
              new_env()),
    {{data, false}, _} =
        apply([eq, [quote, []], [quote, a]],
              new_env()),
    {{data, false}, _} =
        apply([eq, [quote, a], [quote, []]],
              new_env()),
    {{data, false}, _} =
        apply([eq, [quote, [a]], [quote, [b]]],
              new_env()),
    {{data, false}, _} =
        apply([eq, [quote, [a,b,c]], [quote, [a,b,c]]],
              new_env()).
\end{SourceCode}

在Erlang里,可以用 \verb|eq/2| 表示

\begin{SourceCode}[75][81]{func6.erl}
eq([], []) ->
    true;
eq(X, Y)
  when is_atom(X), is_atom(Y) ->
    X =:= Y;
eq(_, _) ->
    false.
\end{SourceCode}


直接用一下Erlang里的 \verb|eq/2| 就定义出 \verb|eq| 来了。

\begin{SourceCode}[50][53]{func6.erl}
call({fn, eq}, [X,Y], Env) ->
    {[{data,X1}, {data,Y1}], Env1} =
        eval_list([X,Y], Env),
    {{data, eq(X1,Y1)}, Env1}.
\end{SourceCode}

\subsection{lambda}

可以用 \verb|lambda| 定义函数。



\subsection{macro}

\begin{Exercise}[title={defun},difficulty=1]
\end{Exercise}



\end{preview-page}
\end{document}
