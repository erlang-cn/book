\documentclass[preview,multi,crop=false,border=1in,class=memoir]{standalone}
\usepackage{../common}

\begin{document}
\begin{preview-page}

\section{软件安装}

\subsection{Erlang}

\subsubsection{在Linux发行版上安装}

运行以下命令即可安装Erlang。

\begin{figure}[hbt]
\centering
\begin{tabular}{|l|l|}
\hline
发行版 & 命令 \\
\hline
Fedora 19/20 & \verb|su -c 'yum install erlang'| \\
\hline
\end{tabular}
\end{figure}

安装后运行 \verb|erl| 命令即可打开Erlang Shell。

\subsubsection{在Windows上安装}

用Windows须自行从Erlang官方网站
(\url{http://www.erlang.org/download.html})下载安装程序。假如下载很慢,
不妨从erlang-users.jp的镜像(\url{http://erlang-users.jp/})下载。

\nonzeroparskip

安装完,以17.3版为例,可以从 ``开始菜单 -> 所有程序 -> Erlang OTP 17
-> Erlang'' 打开Erlang Shell。为方便以后打开Erlang Shell,不妨在开始菜
单里的Erlang图标上点右键``发送到 -> 桌面快捷方式''。在新建出来的快捷方
式的图标上右键点``属性'',在``快捷方式''页把``起始目录''修改成
\verb|%CD%| 。这样,只需要把这个快捷方式复制
到放代码的目录,就可以在那里打开Erlang Shell了。


\subsubsection{Erlang Shell}

打开Erlang Shell后,第一行会显示Erlang的版本信息,第二行是空行,第三行
会显示Shell的版本信息。因为每次打开总是会显示这三行,所以以后都会忽略这
三行。而 \verb|1> | 表示这后面是Shell里的第1次输入。

\begin{Listing}
Erlang/OTP 17 (erts-6.0) [async-threads:10]

Eshell V6.0  (abort with ^G)
1>
\end{Listing}

输入 \verb|io:format("Hello, world!~n").| ,按回车,会显示
\verb|Hello, world!| 。现在会看到 \verb|2> | ,表示这后面是Shell里的第2次
输入。

\begin{ErlangShellSession}
1> io:format("Hello, world!~n").
Hello, world!
ok
2>
\end{ErlangShellSession}

输入 \verb|halt().| ,按回车后,就退出Erlang Shell了。

\begin{Listing}
1> io:format("Hello, world!~n").
Hello, world!
ok
2> halt().
\end{Listing}

\subsubsection{编译运行Erlang代码}

编辑文件 \verb|hello.erl| 。

\SourceCodeInput{hello.erl}

打开Erlang Shell。假如你用的是Windows,请把桌面上的Erlang Shell快捷方式,
复制到 \verb|hello.erl| 所在目录。并用新复制出来的快捷方式打开Erlang
Shell。

\begin{ErlangShellSession}
1> c(hello).
{ok,hello}
2> hello:world().
"Hello, world!"
3>
\end{ErlangShellSession}

\subsection{GNU Emacs}

\subsubsection{在Linux发行版上安装}

运行以下命令即可安装Emacs。

\begin{figure}[hbt]
\centering
\begin{tabular}{|l|l|}
\hline
发行版 & 命令 \\
\hline
Fedora 19/20 & \verb|su -c 'yum install emacs'| \\
\hline
\end{tabular}
\end{figure}

安装后运行 \verb|emacs| 命令即可打开GNU Emacs。

\subsubsection{在Windows上安装}

用Windows须自行从下载Emacs。在
\url{http://www.gnu.org/software/emacs/#Obtaining}上,找到``nearby GNU
mirror'',并点击进入。进入后,找到windows目录,以24.3版为例,选择
\verb|emacs-24.3-bin-i386.zip|。若下载很慢,不妨从中国科学技术大学
开源软件镜像\url{http://mirrors.ustc.edu.cn/gnu/emacs/windows/}下载。

\nonzeroparskip

下载完只需要解压到合适的目录就可以了。找到解压出来的目录下的bin目录,双
击runemacs.exe即可运行。为方便以后打开,不妨在runemacs.exe上点右键``发
送到 -> 桌面快捷方式''。建议和Erlang Shell一样,把启始目录设置成
\verb|%CD%| 。

\nonzeroparskip

用Windows须自行配置Emacs。按The Erlang mode for
Emacs(\url{http://www.erlang.org/doc/apps/tools/erlang_mode_chapter.html})
里Setup on Windows一节的说明设置.emacs文件。若不知道当前设置的HOME环境
变量,可以进入Erlang Shell查看,运行 \verb|os:getenv("HOME").| ,假如结
果是 \verb|false| ,那就是没有设置HOME环境变量。


\end{preview-page}
\end{document}
