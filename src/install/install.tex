\documentclass[preview,multi,crop=false,border=1in,class=memoir]{standalone}
\usepackage{../common}

\begin{document}
\begin{preview-page}

\section{软件安装}

Erlang/OTP建议使用17或更高版本。编译器建议使用Emacs。

\subsection{Erlang}

\subsubsection{Fedora 21}
运行 \verb|yum install erlang|

安装完成后运行 \verb|erl| 命令即可打开Erlang Shell。

\subsubsection{Debian Wheezy}
运行 \verb|apt-get -t wheezy-backports install erlang|
\footnote{只有在wheezy-backports才有Erlang/OTP 17}

安装完成后运行 \verb|erl| 命令即可打开Erlang Shell。

\subsubsection{在Windows上安装}
运行 \verb|choco install erlang|
\footnote{
这需要先安装好 Chocolatey(\url{https://chocolatey.org/})。
也可以自行从Erlang官方网站
(\url{http://www.erlang.org/download.html})下载安装程序。假如下载很慢,
不妨从erlang-users.jp的镜像(\url{http://erlang-users.jp/})下载。
}

安装后运行 \verb|werl| 命令即可打开Erlang Shell。

\subsubsection{Erlang Shell}

打开Erlang Shell后,第一行会显示Erlang的版本信息,第二行是空行,第三行
会显示Shell的版本信息。因为每次打开总是会显示这三行,所以以后都会忽略这
三行。而 \verb|1> | 表示这后面是Shell里的第1次输入。

\begin{Listing}
Erlang/OTP 17 (erts-6.3) [async-threads:10]

Eshell V6.3  (abort with ^G)
1>
\end{Listing}

输入 \verb|io:format("Hello, world!~n").| ,按回车,会显示
\verb|Hello, world!| 。现在会看到 \verb|2> | ,表示这后面是Shell里的第2次
输入。

\begin{ErlangShellSession}
1> io:format("Hello, world!~n").
Hello, world!
ok
2>
\end{ErlangShellSession}

输入 \verb|halt().| ,按回车后,就退出Erlang Shell了。

\begin{Listing}
1> io:format("Hello, world!~n").
Hello, world!
ok
2> halt().
\end{Listing}

\subsubsection{编译运行Erlang代码}

编辑文件 \verb|hello.erl| 。

% SNIP REFERENCE hello.erl hello-world
\begin{SourceCode}
% SNIP BEGIN hello-world
-module(hello).

-export([world/0]).

world() -> "Hello, world!".
% SNIP END
\end{SourceCode}

打开Erlang Shell。

\begin{ErlangShellSession}
1> c(hello).
{ok,hello}
2> hello:world().
"Hello, world!"
3>
\end{ErlangShellSession}

\subsection{GNU Emacs}

\subsubsection{Fedora 21}
运行 \verb|yum install emacs|

\subsubsection{Debian Wheezy}
运行 \verb|apt-get install emacs|

\subsubsection{Windows}
运行 \verb|choco install emacs|
\footnote{
这需要先安装好 Chocolatey(\url{https://chocolatey.org/})。也可以自行下
载Emacs。在\url{http://www.gnu.org/software/emacs/\#Obtaining}上,找到
``nearby GNU mirror'',并点击进入。进入后,找到windows目录,以24.4版为
例,选择 emacs-24.4-bin-i686-pc-mingw32.zip 。若下载很慢,不妨从
中国科学技术大学开源软件镜像
\url{http://mirrors.ustc.edu.cn/gnu/emacs/windows/}下载。
}

\nonzeroparskip

用Windows须自行配置Emacs。按The Erlang mode for
Emacs(\url{http://www.erlang.org/doc/apps/tools/erlang_mode_chapter.html})
里Setup on Windows一节的说明设置 \verb|.emacs| 文件。若不知道当前设置的HOME环境
变量,可以进入Erlang Shell查看,运行 \verb|os:getenv("HOME").| ,假如结
果是 \verb|false| ,那就是没有设置HOME环境变量。

\end{preview-page}
\end{document}
